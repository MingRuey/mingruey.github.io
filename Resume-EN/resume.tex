% # -*- coding:utf-8 -*-
%% start of file `resume.tex'.
%% Author imchou239@gmail.com 
%
% Copyright 2006-1008 Xavier Danaux (xdanaux@gmail.com).
% This work may be distributed and/or modified under the
% conditions of the LaTeX Project Public License version 1.3c,
% available at http://www.latex-project.org/lppl/.

\documentclass[11pt,a4paper]{moderncv}

\usepackage{fontspec}
\setmainfont{Roboto}
\usepackage{xcolor}

\setmainfont{Roboto}
\XeTeXlinebreakskip = 0pt plus 1pt minus 0.1pt

% moderncv themes
\moderncvtheme[orange]{classic}                 
% optional argument: 'blue' (default), 'orange', 'red', 'green', 'grey' and 'roman' (for roman fonts, instead of sans serif fonts)

% adjust the page margins
\usepackage[scale=0.9]{geometry}

% change the colmn width of dates
%\setlength{\hintscolumnwidth}{3cm}

% Classic theme only, change the width of name placeholder 
% (to leave more space for your address details
%\AtBeginDocument{\setlength{\maketitlenamewidth}{6cm}}
\AtBeginDocument{\recomputelengths}

% personal data
\firstname{Ming-Ruey(Ray)}
\familyname{Chou}

% optional
\title{  }
  %\address{1990/09/11}{}
  %\fax{fax}
\mobile{(+886) 966525557}
\address{}{}
\email{imchou239@gmail.com}
  %\homepage{Blog: http://geekplux.com}
\social[github]{MingRuey}
\extrainfo{%
  Twitter: imchou239\\
  6F, No.3, Ln.125, Yitong St., Taipei, Taiwan
}
% \quote{“There is no authority who decides what is a good idea.”}
% \photo[55pt]{RayModified_Cropped.jpg}

% Suppress automatic page numbering for CVs longer than one page
%\nopagenumbers{}


%----------------------------------------------------------------------------------
%            content
%----------------------------------------------------------------------------------
\begin{document}
\maketitle
\vspace*{-18mm}

\section{SKILLLS}
\cvline{\textbf{Programming}}{Python:\space\footnotesize{Linux - IDE \& Virtualenv/Docker, Windows - Anaconda},\quad\normalsize Java,\quad IDL,\quad Labview}
\cvline{\textbf{\footnotesize Machine Learning}}{\vspace*{0cm}TensorFlow,\quad Keras, \quad Scikit-Learn, \quad Pandas,\quad Lightgbm}
\cvline{\textbf{Visualization}}{Matplolib,\quad Seaborn;\quad OriginLab}
\cvline{\textbf{Database}}{SQL \& MySQL}
\cvline{\textbf{Language}}{Mandarin:\space\space Native;\quad English:\space\space Fluent}
\vspace{-0.7\baselineskip}

\section{PROJECTS}
\subsection{\href{https://www.kaggle.com/c/imaterialist-challenge-fashion-2018}{Kaggle - iMaterialist Challenge(Fashion) at FGVC5}}
\cvline{\footnotesize{Top 16\%}}{\footnotesize{Multi-label classification competition over one million images. I did transfer learning for standard CNNs with Keras, and use GrabCut and spectral saliency detection to segment background/foreground for preprocessing.}}
\subsection{\href{https://www.kaggle.com/c/avito-demand-prediction}{Kaggle - Avito Demand Prediction Challenge}}
\cvline{\footnotesize{Top 27\%}}{\footnotesize{Predict the demand over one million products on Avito.ru website. After standard missing value handling, one-hot encoding, and data normalization, I tackled the task with iterative process of EDA and GBDT modeling with LightGBM. Using both feature importance given by the model and observation from EDA to improve the model.}}
\subsection{\href{https://www.kaggle.com/c/google-ai-open-images-object-detection-track}{Kaggle - Google AI Open Images - Object Detection Track}}
\cvline{\footnotesize{Top 19\%}}{\footnotesize{An object detection competition on \href{https://storage.googleapis.com/openimages/web/index.html}{Open Images Dataset V4}. I packed the data into TFRecord format and used \href{https://github.com/tensorflow/models/tree/master/research/object_detection}{TensorFlow object detection API} to do transfer learning on Faster-RCNN architecture.}}
\subsection{\href{https://www.kaggle.com/c/rsna-pneumonia-detection-challenge}{Kaggle - RSNA Pneumonia Detection Challenge}}
\cvline{\footnotesize{Top 21\%}}{\footnotesize{Predict the location of Pneumonia on X-ray images. Using TensorFlow, I built Unet and Faster-RCNN from scratch, trained models with Estimator API and applied data augmentation including half-splitting, rotating, and flipping.}}
\subsection{\href{https://tbrain.trendmicro.com.tw/Competitions/Details/2}{T-Brain AI- Taiwan ETF Price Prediction}}
\cvline{}{\footnotesize{Use ARIMA to predict the time series of ETF prices. Also experiemt with traditional technical indicators like RSI, stochastic oscillator, \%R, Chaikin oscillator, CCI .etc as features for a random forest classifier.}}
\vspace{-0.7\baselineskip}

\section{EDUCATION}
% \cventry{year--year}{Job title}{Employer}{City}{}{Description}
% arguments 3 to 6 are optional
\cventry{\scriptsize{SEP, 13 - JUL, 16}}{M.S. in Physics}{Department of Physics}{National Taiwan University}{}{}
\cventry{\scriptsize{SEP, 09 - JUL, 16}}{B.S. in Physics}{Department of Physics}{National Taiwan University}{}{}
\cventry{}{Course Work -- Coursera :}{}{}{}{}
\cvlistitem{Algorithms 4E, Robert Sedgewick and Kevin Wayne; Princeton}
\cvlistitem{Machine Learning Foundations, Hsuan-Tien Lin; NTU}
\cvlistitem{Machine Learning, Andrew Ng; Standford}
\cvlistitem{Introduction to SQL, Charles Severance; Michigan}
\cvlistitem{SQL for Data Science, Sadie St. Lawrence; UC Davis}
\vspace{-0.7\baselineskip}

\section{WRITING}
\cvline{\footnotesize{M.S. Thesis}}{\textbf{Rheometry on Concentrated Suspension of Soft Particles}}{}
\cvline{}{\href{http://www.phys.sinica.edu.tw/jctsai/Ray2016/}{(In Mandarin) \emph{www.phys.sinica.edu.tw/jctsai/Ray2016/}}}
\vspace{-0.7\baselineskip}

\section{OTHER EXPERIENCE}
\cventry{\scriptsize{JAN, 18 ; JUL, 18}}{Twin Oaks Education}{Course Design and Project Mentor}{\href{http://www.twinoaks-edu.com/}{www.twinoaks-edu.com/}}{}{}
\cventry{\scriptsize{SEP, 16 - AUG, 17}}{Substitue Services in Education}{Xinyi Elementary School}{
Hualien, Taiwan}{}{}


% \section{Computer skills}
% \cvcomputer{category 1}{XXX, YYY, ZZZ}{category 4}{XXX, YYY, ZZZ}
% \cvcomputer{category 2}{XXX, YYY, ZZZ}{category 5}{XXX, YYY, ZZZ}
% \cvcomputer{category 3}{XXX, YYY, ZZZ}{category 6}{XXX, YYY, ZZZ}

% \section{Interests}
% \cvline{橋牌}{\small 沒有高手的程度 也要有高手的風度}
% \cvline{桌遊}{\small 買了一屋子遊戲 找不到一桌朋友}
% \cvline{鋼琴}{\small 誰叫你不練琴呢?}

% \renewcommand{\listitemsymbol}{-} % change the symbol for lists

% \section{Extra 1}
% \cvlistitem{Item 1}
% \cvlistitem{Item 2}
% \cvlistitem[+]{Item 3}            

% \section{Extra 2}
% \cvlistdoubleitem[\Neutral]{Item 1}{Item 4}
% \cvlistdoubleitem[\Neutral]{Item 2}{Item 5}
% \cvlistdoubleitem[\Neutral]{Item 3}{}

%% Publications from a BibTeX file
% \nocite{*}
% \bibliographystyle{plain}
% \bibliography{publications}
% \begin{thebibliography}{}
% \bibitem[]{}
% \end{thebibliography}


\end{document}


%% end of file `resume.tex'.

%%% Local Variables:
%%% mode: latex
%%% TeX-command-extra-options: "-shell-escape"
%%% TeX-master: t
%%% TeX-engine: xetex
%%% End: